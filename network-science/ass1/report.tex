\documentclass[12pt]{article}

\begin{document}
\title{COMP 767 - Assignment 1}
\author{Yi Tian Xu}
\date{September 15, 2016}
\maketitle

\section{Vintage Luxury Trends}

One dataset that can interesting and challenging to look at comes from the company that I worked with in the past year.


\section{Phylogenetic Tree}

Motived by my interest in evolution theory, I choose the phylogenetic tree for my second network. Defining species as vertices and evolutionary relationships as edges, we can obtain a directed graph. We can call a species $u$ a parent of species $v$ if species $v$ is evolved from species $u$ and no other species is in the evolution path between $u$ and $v$. This relationship can be marked by a directed edge from child to parent. Some potentially interesting exercises that can be done with this network are visualization and clustering (e.g.: classifying species into families). 

\subsection{Data Source and Gathering Method}

The data I gathered is from the Ensembl Rest API (https://rest.ensembl.org). Their taxonomic classification endpoint returns a list of parents and each of their children for a queried species. 

The method that I used is a script that queries for 


\end{document}